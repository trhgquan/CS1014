\documentclass[12pt]{article}

\usepackage{amsmath}
\usepackage{amsfonts}
\usepackage[affil-it]{authblk}
\usepackage{float}
\usepackage{fancyhdr}
\usepackage{graphicx}
\usepackage[colorlinks=true,linkcolor=blue, citecolor=red]{hyperref}
\usepackage{url}
\usepackage[top=.75in, left=.5in, right=.5in, bottom=1in]{geometry}
\usepackage[utf8]{vietnam}
\setlength{\headheight}{29.43912pt}

% \graphicspath{PATH_TO_GRAPHIC_FOLDER}

\newcommand{\coursename}{CSC10104 - Quy hoạch tuyến tính}
\newcommand{\reportname}{Một ví dụ cho bài toán Klee-Minty với $n = 4$}

\pagestyle{fancy}
\lhead{
\reportname
}
\rhead{
Trường Đại học Khoa học Tự nhiên - ĐHQG HCM\\
\coursename
}
\lfoot{\LaTeX\ by \href{https://github.com/trhgquan}{Quan, Tran Hoang}}

\title{\textbf{\reportname}}
\author{Trần Hoàng Quân%
  \thanks{MSSV: 19120338; Email: \texttt{19120338@student.hcmus.edu.vn}}}
\affil{Trường Đại học Khoa học Tự nhiên, ĐHQG-HCM}

\author{Đoàn Thu Ngân%
  \thanks{MSSV: 19120xxx; Email: \texttt{19120xxx@student.hcmus.edu.vn}}}
\affil{Trường Đại học Khoa học Tự nhiên, ĐHQG-HCM}


\begin{document}
\maketitle

\begin{abstract}
Trong bài này, nhóm nhắc lại định nghĩa về Bài toán Klee-Minty \ref{klee-minty-revised}. Để đưa ra một ví dụ minh họa, nhóm chọn Bài tập 4.5 từ quyển Linear Programming của tác giả Robert J. Vanderbei \ref{Vanderbei-4.5}. Qua đó, nhóm đưa ra kết luận rằng bài toán Klee-Minty cần ít nhất $2^n - 1$ lần lặp để tìm ra được phương án tối ưu, với $n$ là số ràng buộc và số biến.
\end{abstract}

\section{Nhắc lại bài toán Klee-Minty}\label{klee-minty-revised}
Bài toán Klee-Minty là một bài toán quy hoạch tuyến tính có $n$ ràng buộc và $n$ biến, được đưa ra bởi V.Klee và G.J. Minty vào năm 1972\cite{Vanderbei2020}. Theo đó, để giải bài toán này dùng Phương pháp đơn hình cần ít nhất $2^{n} - 1$ lần lặp (hay nói cách khác, cần kẻ $2^{n} - 1$ bảng đơn hình để tìm ra phương án tối ưu). Bài toán được phát biểu như sau:\\\\
Hàm mục tiêu
$$
f(x_1, .., x_n) = \sum_{j = 1}^n 2^{n - j}x_j \rightarrow \max
$$
các ràng buộc
\begin{equation*}
\begin{aligned}
2\sum_{j = 1}^{i - 1} 2^{i - j}x_j + x_i &\leq 100^{i - 1}\ &i = \overline{1, n}\\
x_j &\geq 0\ &j = \overline{1, n}
\end{aligned}
\end{equation*}
Tổng quát hóa: đặt $\beta_i = 100^{i - 1}$. Khi đó dạng tổng quát hóa của Bài toán Klee-Minty trở thành:\\\\
Hàm mục tiêu
$$
\sum_{j = 1}^n 2^{n - j}x_j - \frac{1}{2}\sum_{j = 1}^{n} 2^{n - j}\beta_j \rightarrow \max
$$
các ràng buộc
\begin{equation*}
\begin{aligned}
\sum_{j = 1}^{i - 1} 2^{i - j} x_j + x_i &\leq \sum_{j = 1}^{i - 1} 2^{i - j} \beta_j + \beta_i\ &i = \overline{1, n}\\
x_j &\leq 0\ &j = \overline{1, n}
\end{aligned}
\end{equation*}
Bài toán Klee-Minty là ví dụ cho trường hợp xấu nhất của Phương pháp đơn hình, khi đó ta tốn nhiều chi phí tính toán nhất để giải quyết một bài toán.

\section{Bài tập 4.5 sách Vanderbei}\label{Vanderbei-4.5}
\subsection{Đề bài}
Giải bài toán Klee-Minty với $n = 4$. Với trường hợp $n = 4$, hàm mục tiêu trở thành
$$
f(x_1, x_2, x_3, x_4) = 8x_1 + 4x_2 + 2x_3 + x_4 - 4\beta_1 - 2\beta_2 - \beta_3 - \frac{1}{2}\beta_4 \rightarrow \max
$$
các ràng buộc
$$
\left\{
\begin{array}{lll}
x_1 &\leq \beta_1 \\
4x_1 + x_2 &\leq 2\beta_1 + \beta_2 \\
8x_1 + 4x_2 + x_3 &\leq 4\beta_1 + 2\beta_2 + \beta_3 \\
16x_1 + 8x_2 + 4x_3 + x_4 &\leq 8\beta_1 + 4\beta_2 + 2\beta_3 + \beta_4 \\
x_j &\geq 0\ &j = \overline{1, 4}
\end{array}
\right.
$$

\subsection{Lời giải}
Bổ sung các biến $x_5, x_6, x_7, x_8$ để biến hệ ràng buộc về dạng đẳng thức:
$$
\left\{
\begin{array}{lll}
x_1 + x_5 &= \beta_1 \\
4x_1 + x_2 + x_6 &= 2\beta_1 + \beta_2 \\
8x_1 + 4x_2 + x_3 + x_7 &= 4\beta_1 + 2\beta_2 + \beta_3 \\
16x_1 + 8x_2 + 4x_3 + x_4 + x_8 &= 8\beta_1 + 4\beta_2 + 2\beta_3 + \beta_4 \\
x_j &\geq 0\ &j = \overline{1, 4}
\end{array}
\right.
$$
Bảng đơn hình xuất phát như sau:
\begin{table}[H]
\centering
\begin{tabular}{|c|c|c|c|c|c|c|c|c|c|c|}
\hline
CS & HS & PA & 8 & 4 & 2 & 1 & 0 & 0 & 0 & 0 \\
\hline
$x_5$ & 0 & $\beta_1$ & 1 & 0 & 0 & 0 & 1 & 0 & 0 & 0 \\
$x_6$ & 0 & $2\beta_1 + \beta_2$ & 4 & 1 & 0 & 0 & 0 & 1 & 0 & 0 \\
$x_7$ & 0 & $4\beta_1 + 2\beta_2 + \beta_3$ & 8 & 4 & 1 & 0 & 0 & 0 & 1 & 0 \\
$x_8$ & 0 & $8\beta_1 + 4\beta_2 + 2\beta_3 + \beta_4$ & 16 & 8 & 4 & 1 & 0 & 0 & 0 & 1 \\
\hline
\multicolumn{2}{|c|}{$f_{\max}$}
& 0 & -8 & -4 & -2 & -1 & 0 & 0 & 0 & 0 \\
\hline
\end{tabular}
\end{table}

\subsubsection{Lần lặp thứ 1}
\begin{itemize}
\item Nhận thấy có 3 cột có $\Delta < 0$, ta chọn cột có $\Delta = -8$.
\item Xét tỷ lệ $\displaystyle \frac{\beta_1}{1} < \frac{2\beta_1 + \beta_2}{4} < \frac{4\beta_1 + 2\beta_2 + \beta_3}{8} < \frac{8\beta_1 + 4\beta_2 + 2\beta_3 + \beta_4}{16}$, nên ta chọn dòng 1.
\item Khi đó $x_5$ ra, $x_1$ vào.
\end{itemize}
Tiến hành biến đổi:
\begin{itemize}
\item $d_1 \leftarrow d_1$
\end{itemize}
Bảng đơn hình cho lần lặp đầu tiên như sau:
\begin{table}[H]
\centering
\begin{tabular}{|c|c|c|c|c|c|c|c|c|c|c|}
\hline
CS & HS & PA & 8 & 4 & 2 & 1 & 0 & 0 & 0 & 0 \\
\hline
$x_1$ & 8 & $\beta_1$ & 1 & 0 & 0 & 0 & 1 & 0 & 0 & 0 \\
$x_6$ & 0 & $2\beta_1 + \beta_2$ & 4 & 1 & 0 & 0 & 0 & 1 & 0 & 0 \\
$x_7$ & 0 & $4\beta_1 + 2\beta_2 + \beta_3$ & 8 & 4 & 1 & 0 & 0 & 0 & 1 & 0 \\
$x_8$ & 0 & $8\beta_1 + 4\beta_2 + 2\beta_3 + \beta_4$ & 16 & 8 & 4 & 1 & 0 & 0 & 0 & 1 \\
\hline
\multicolumn{2}{|c|}{$f_{\max}$}
& $8\beta_1$ & -8 & -4 & -2 & -1 & 0 & 0 & 0 & 0 \\
\hline
\end{tabular}
\end{table}

\cleardoublepage
\phantomsection
\addcontentsline{toc}{section}{Tài liệu}
\bibliographystyle{plain}
\bibliography{sample}
\end{document}