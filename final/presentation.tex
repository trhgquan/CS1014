\documentclass[10pt]{beamer}
\usepackage[utf8]{vietnam}
%\usepackage[T1]{vietnam}
\usepackage{lmodern}
\usetheme{Darmstadt}

\usepackage[backend=biber]{biblatex}

\addbibresource{presentation.bib}

\begin{document}
\author{Đoàn Thu Ngân, Trần Hoàng Quân, Huỳnh Tấn Thọ, Sử Nhật Đăng, Phan Đặng Diễm Uyên}
\title{Efficiency of the Simplex Method}
\subtitle{Investigating on how fast Simplex Method will solve a problem in a given size.}
\logo{img/fithcmuslogo.png}
\institute{VNUHCM - University of Science}
\date{Spring 2022}
\subject{CSC10104 - Linear Programming}

%\setbeamercovered{transparent}
\setbeamertemplate{navigation symbols}{}

\begin{frame}[plain]
\maketitle
\end{frame}

\begin{frame}
\frametitle{Table of Contents}
\tableofcontents
\end{frame}

\section{Introduction}
\begin{frame}{Introduction}

\end{frame}

\section{Performance Measures}
\begin{frame}{Performance Measures}
Can be divied into two types:
\begin{itemize}
\item Worst case: asking how much efforts is needed to solve all problems of a given \textit{size}
\item Average case: looking at the average amount of effort, averaging over all problems of a giving \textit{size}. 
\end{itemize}
Worst-case analyses are generally easier than Average-case.
\end{frame}

\section{Measuring the Size of a Problem}
\begin{frame}{Measuring the Size of a Problem}

\end{frame}

\section{Measuring the Effort to Solve a Problem}
\begin{frame}{Measuring the Effort to Solve a Problem}

\end{frame}

\section{Worst-case Analysis of the Simplex Method}
\begin{frame}{Worst-case Analysis of the Simplex Method}
The upper bound of basic feasible solutions is $\displaystyle {n + m \choose n}$. For a fixed value of the sum $n + m$, this expression is maximized when $m = n$.

Hence, we can easily estimate an upper and lower bound of $\displaystyle {2n \choose n}$:

$$
\frac{4^{n}}{2n} \leq {2n \choose n} \leq 4^{n} \ (n \in \mathbb{N^+})
$$
\end{frame}
\begin{frame}{Worst-case Analysis of the Simplex Method (cont.)}
\begin{theorem}[Stirling's approximation]
$$
\displaystyle
n! \operatorname*{\sim}_{n\to\infty} \sqrt{2\pi n}\left(\frac{n}{e}\right)^n
$$
\end{theorem}
Given that $\displaystyle {2n \choose n} = \frac{(2n)!}{(n!)^2}$ and following expressions:
\begin{itemize}
\item $\displaystyle (2n)! \operatorname*{\sim}_{n\to\infty} 2\sqrt{\pi n} \left(\frac{2n}{e}\right)^{2n}$ 
\item $\displaystyle (n!)^2 \operatorname*{\sim}_{n\to\infty} 2\pi n \left(\frac{n}{e}\right)^{2n}$ 
\end{itemize}
Hence, $\displaystyle {2n \choose n} \operatorname*{\sim}_{n\to\infty} \frac{2\sqrt{\pi n} \left(\frac{2n}{e}\right)^{2n}}{2\pi n \left(\frac{n}{e}\right)^{2n}} =  \frac{4^n}{\sqrt{\pi n}}$
\end{frame}

\begin{frame}{Worst-case Analysis of the Simplex Method (cont.)}
\begin{proof}
Let
$$
\displaystyle
L = \frac{\frac{4^n}{2n}}{\frac{4^n}{\sqrt{\pi n}}} = \frac{4^n\sqrt{\pi n}}{4^n 2n} = \frac{\sqrt{\pi}}{2} \frac{1}{\sqrt{n}} < 1\ \forall n \in \mathbb{N^+}
$$
Since $L < 1$, we can state that $\frac{4^n}{2n} \leq \frac{4^n}{\sqrt{\pi n}},\ \forall n \in \mathbb{N^+}
$
. Hence,
$$
\displaystyle
\frac{4^n}{2n} \leq \frac{4^n}{\sqrt{\pi n}} \leq 4^n \iff \frac{4^n}{2n} \leq {2n \choose n} \leq 4^n
$$
\end{proof}
This states that by the worst case, the algorithm has the complexity of $\Theta(4^n)$.
\end{frame}

\section{Empirical Average Performance of the Simplex Method}
\begin{frame}{Empirical Average Performance of the Simplex Method}

\end{frame}

\section{References}
\begin{frame}[allowframebreaks]{References}
\printbibliography
\end{frame}
\end{document}